\documentclass[12pt,a4paper]{article}
\usepackage[utf8]{inputenc}
%\usepackage[a4paper, margin = 1in]{geometry}
\usepackage[a4paper]{geometry}
%\usepackage[none]{hyphenat}

\nocite{*}

\usepackage[
backend=biber,
style=apa,
sorting=apa
]{biblatex}

\addbibresource{sgs.bib}

\title{Inheritance of a tragedy \footnote{Originally written for the Global Essay Competition 2023 at St. Gallen Symposium}}
\author{Saptarshi Chakrabarti}
\date{1st February 2023}

\begin{document}

\maketitle

\begin{abstract}
The relationship between climate change and economic growth in developing nations is examined in this essay. It draws attention to the difficult conundrum these nations face in balancing economic growth and environmental sustainability. The essay makes the case that the development of developing countries' economies has frequently come at the expense of their environments due to the inherited abundance of resources and resource-related sectors. This has brought their interest in survival into conflict with the shift to a low-carbon economy, which is contrary to their objectives of economic growth and inequality reduction. The situation is seen in the essay as a multi-player Prisoner's Dilemma in which the economies of the developing world are the players, and Pareto-efficient solutions are rarely found. The essay acknowledges the risks associated with the economic growth that is driven by energy sources that the current generation has inherited, risks that will only result in a broken world for future generations. The essay also points out that actions taken to combat climate change frequently contribute to the very tragedy of the commons that they are meant to address. The essay suggests a solution to this conundrum: a combination of policies with regular evaluations of their effectiveness will help everyone come to the table and be motivated to work together, resulting in more stable and long-lasting solutions for a healthier future. In conclusion, this essay sheds light on the complex interplay between climate change and economic development in developing countries and underscores the need for policies that balance economic growth and environmental sustainability.
\end{abstract}

\newpage
\section*{Introduction}
Climate change is one of the most pressing issues of our time, and it affects all countries, developed or developing. However, the challenge is particularly acute for developing countries, which often face the dilemma of balancing economic growth and environmental sustainability. In this essay, I argue that economic growth in developing countries is primarily credited to an inherited abundance of resources and sectors linked to such resources. Such growth has come at the cost of their environmental conditions – thus tying up their interest in survivability with the transition to a low-carbon economy. However, the risks associated with such transitions create conflicts with their goal of economic development and reducing inequality. The situation can be interpreted as a multi-player Prisoner's dilemma where the economies of the developing world are the players of the game, and the solutions are seldom Pareto-efficient. 

\section*{Inheritance of an energy-fueled legacy of economic growth}
Developing countries have inherited an energy-fuelled legacy of economic growth due to their abundant fossil fuel resources, such as oil and gas, and the expansion of energy-intensive industries. One major factor is the abundance of fossil fuel resources in many developing countries, which has allowed for the development of industries such as oil and gas production and power generation, providing a significant source of economic growth. Venezuela's economy is dominated by oil and constitutes over 90\% of its exports and 15-25\% of its GDP (Gualdron \& Manzano, 2022). Russia heavily relies on oil and gas revenue, making up 45\% of its federal budget (Energy Fact Sheet, 2022). Saudi Arabia considers the energy sector the backbone of its economy (Energy | The Embassy of The Kingdom of Saudi Arabia, n.d.), with oil company Aramco reporting a 90\% YoY increase in profit (`Saudi Aramco', 2022). In Indonesia, the energy sector is crucial to its economic growth and is based on natural resource extraction, with coal being the primary export (Asian Development Bank, 2021). Brazil is one of the top 10 oil producers in the world and has doubled its primary energy demand since 1990 due to strong growth in electricity and transport fuels (Angelos Delivorias, 2022).
Additionally, the increasing demand for energy worldwide, particularly in rapidly industrialising countries, has expanded these industries in developing countries. Moreover, the availability of affordable and reliable energy has enabled the growth of other industries in developing countries, such as manufacturing, agriculture and transportation, which are energy-intensive sectors. Expanding these industries has led to increased employment opportunities and economic growth. 

\newpage
\section*{Environmental cost \& increased dependence on foreign energy sources}
Notably, this growth has been accompanied by negative environmental impacts and an increasing dependence on foreign energy sources. In India, dependence on foreign oil has made it susceptible to supply disruptions and price hikes, with a 10\% increase in oil prices subtracting 0.2\% from GDP growth (Bundhun, 2022). China's rapid growth fueled by coal has led to high air pollution and environmental degradation. Political instability in the Middle East represents a significant energy security concern, as 50\% of its oil imports come from the region (`How Is China's Energy Footprint Changing?', 2016). Coal production is one of the critical drivers of the economy in South Africa. However, the growing environmental concerns and financial distress of Eskom, the largest producer of electricity in Africa, which accounts for over 40\% of emissions and has a debt of over \$27 billion, are leading to calls for a more sustainable energy source, considering poverty, unemployment, and the number of workers relying on the coal industry (Saliem Fakir, 2021). Nigeria, a significant oil producer, faces environmental catastrophe due to oil extraction, resulting in contaminated air, land, and water, negative impacts on health and livelihoods, and increased criminal activities. Studies have shown that oil spills and gas flares have devastated communities relying on fishing or farming. However, operating companies have blamed illegal activities such as crude oil theft, illegal refining, and sabotage for most environmental damage (Ratcliffe, 2019). 
\section*{Debt \& Transition Risk: Dark Side of Energy-Intensive Growth in Developing Countries}
Reliance on energy-intensive growth for development exposes countries to high levels of financial debt and climate transition risk. These countries often accumulate large debt, rely on exporting fossil fuels, and have limited access to international financial support, all contributing to the risk. Additionally, the financial debt can prevent them from accessing funding for climate change adaptation and mitigation measures, making them more vulnerable to its impacts. Developing countries like Venezuela, Indonesia, South Africa, and Nigeria are facing the impact of relying on energy-intensive economic growth on their financial debt and climate transition risk. Venezuela is a failed petrostate, with oil accounting for 99\% of export earnings and 25\% of GDP. The country is facing high debt (\$150 billion or more) and hyperinflation (1,946\%) due to the lack of investment and maintenance in the oil industry, leading to a severe humanitarian crisis (Amelia Cheatham et al., 2021). Indonesia is a significant coal producer and exporter, with coal mining accounting for 8\% of its GDP (Government of Indonesia, 2022). Despite providing employment, relying on coal raises concerns about financial debt and climate transition risk. South Africa is facing financial debt and climate transition risk due to its reliance on coal in the energy and mining sectors, but there is a growing recognition of the need for climate change mitigation, leading to conflicts between actors advocating for a low-carbon transition and those connected to coal (Hanto et al., 2022). Nigeria's reliance on oil exports has led to high financial debt and climate transition risk, with limited international financing for adaptation and mitigation measures. Despite being a recipient of climate finance, it remains insufficient to achieve Nigeria's NDC, with a gap of USD 15.8 billion annually. When countries struggle to manage their existing debt, they may be reluctant to take on additional debt to finance the transition to a low-carbon economy (Meattle, 2022). This is especially true if the expected benefits of this transition are uncertain or seen as uncertain. 
\section*{Dilemma of Developing Economies: Trapped in a No-Win Situation}
Developing countries are facing a difficult situation, caught in the middle of two conflicting priorities. On the one hand, they need to promote economic growth to lift their citizens out of poverty and provide essential government services. On the other hand, they must also address the urgent issue of climate change and its impacts on their citizens and their environment. The problem is that both objectives can be achieved only if all countries collectively take action to mitigate the negative impacts of climate change, but each country faces conflicting incentives to prioritise economic growth over environmental protection. We are then left with a ``prisoner's dilemma"  - an inheritance with no upsides.  \par
The problems of poverty and slow economic growth are interlinked in developing countries. Lack of job opportunities, low foreign investment and trade, and a decline in tax revenue are just a few of the consequences of slow economic growth. This, in turn, leads to a lack of purchasing power for the citizens, further slowing economic growth and creating a vicious cycle. The government is also unable to provide essential services such as healthcare and education, exacerbating the problem. \par
The situation is made even more challenging by the need for developing countries to address the issue of climate change. Financial and technical assistance from industrialised countries, which have contributed the most to climate change, is essential. However, the lack of resources, focus on economic development, dependence on fossil fuels and other pressing issues such as poverty, hunger, and access to clean water can make it difficult for developing countries to implement strong climate policies. \par
The planet's resources, such as the atmosphere and its ability to regulate temperature, are used collectively by all countries. However, with each country acting in self-interest, they are overused and depleted, leading to negative consequences for everyone. This is a classic example of a ``tragedy of the commons," where the resources shared by all are depleted due to each individual acting in their own self-interest.
\section*{The Climate Infinite Game}
Policy-based strategies for governments and private entities are one way of dealing with this. Climate change is an infinite game in which future generations will be affected by our choices today. As a result, the public must prioritise long-term payoffs in this game. One way to do this is by putting an economic cost on environmental damage. This means incorporating the negative impacts of carbon emissions and other pollutants on the environment into the cost of goods and services. Economic costs motivate politicians in economies where climate transition is otherwise not a priority- to campaign for climate change and pave the way for public discourse on the significance of it. Implementing a carbon tax or cap-and-trade system, where companies and individuals pay for their carbon emissions, makes environmental degradation visible and gives companies a financial incentive to reduce their carbon footprint. The cost of emissions would also generate revenue for governments to fund climate policies, such as investment in clean energy or public transportation. \par
For private sector entities, the terms of debt can be altered to encourage low-carbon solutions. Banks and financial institutions can be motivated to provide low-cost loans to companies investing in renewable energy or energy-efficient technologies. Additionally, green bonds can be established, where companies or governments issue debt securities to finance eco-friendly projects. This provides private entities with financial incentives as positive reinforcement to transition to low-carbon solutions, supporting a more sustainable future.   \par
Policymakers can also incorporate the social cost of carbon – an economic estimate of the damages created by one extra ton of carbon dioxide emissions. By placing a monetary value on carbon emissions, policymakers can consider these costs in their climate policy decisions, aligning their interests with the environment and promoting sustainable economic growth.  \par
Changing the objectives and payoffs in the prisoner's dilemma can motivate policymakers and private entities to address climate change and promote a sustainable future. However, policies aimed at tackling climate change in developing countries can give rise to a free-rider problem, where some nations may avoid taking action to reduce their carbon emissions because they believe that other countries will bear the cost. This leads us back to the tragedy of the commons, where everyone shares the benefits of the global environment, but only a few bear the costs. This free-rider problem can fail to achieve the desired level of global cooperation in reducing carbon emissions, leading to continued degradation of the global environment.  \par
Climate agreements, which are frequently used to encourage international cooperation, are frequently non-binding, meaning that nations are not obligated to uphold their commitments. This may make it challenging for nations to hold one another responsible for their deeds and erode international trust.  \par
As a result, some nations might be hesitant to sign up for these agreements or, even if they do, might not stick to their promises. The fact that climate agreements frequently rest on people's goodness and foresight makes them unstable. While doing so can be an excellent way to promote cooperation, it can also be challenging to maintain over time. \par
Legally binding climate accords that incentivise cooperation by establishing a system of accountability and removing barriers to the transition to low-carbon economies are necessary to address the challenge of climate change. Countries can ensure that their actions align with their stated goals and take the necessary steps to transition to low-carbon economies by establishing legally binding commitments. Furthermore, by removing transition barriers such as limiting access to resources, countries can ensure that all nations have the necessary resources to make this transition, reducing the risk of free-riding and increasing the overall effectiveness of the agreement.
\section*{Conclusion}
%Policies generally targeted towards the problems of climate change often feed back to the same tragedy of commons that it seeks to alleviate. A combination of policies with frequent evaluations of their efficacy targeted towards bringing everyone to the table and incentivising them to cooperate goes a long way in creating stable solutions towards a more sustainable and healthy future.
The inheritance of economic growth in developing countries, fueled by energy sources and the abundance of resources and sectors linked to such resources, has resulted in a tragedy of the commons. The situation presents itself as a legacy of environmental degradation and risks associated with the current generation's pursuit of economic growth, which will only result in a broken world for future generations. The essay suggests that a combination of policies with regular evaluations of their effectiveness is necessary to bring everyone to the table and incentivise cooperation towards a more sustainable future. The complexity of the interplay between climate change and economic development in developing countries highlights the urgency of finding a balance between these two factors. This essay serves as a call to action for policymakers to address this critical issue, and I hope that it will contribute to the global effort to create a healthier and more sustainable future for all.

\printbibliography
\end{document}
